\input setbmp
\section{CHAPTER FOUR:The Crime Investigation Tool}

\noindent In this chapter we discuss the building blocks of the crime investigation tool, its operation and functionality and finally present the test results based on the parameters presented in in the previous chapter.


\subsection{Crime Events and Parameters}

\noindent The system was built to support five events representing the investigation pillars for crimes realted to causing financial loss to government(Funds Request, Funds Release, Payments, Phone Logs, Email Logs). The tool provides an interface for users to define criteria for each of the five events to instruct the CEP engine on which fields and values to check.

[Add diagram for flow]

\subsubsection{Funds Request By Ministry}

\noindent Before any amount of money is disbursed to any government entity there must be a budget designed at the Ministry of Finance (government) stipulating funds available for different activities. These funds are availed on a periodic basis to the entity most commonly quarterly.

\noindent Likewise the government entity requesting for funds must provide a detailed budget to the Ministry of Finance if it is another ministry or through the Ministry of local government for Local Governments. This work plan provides a form of accountability to government for the released amounts.Requests for release of funds must duly be authorized by a designated official permitted to act on behalf of the entity.

\noindent Therefore for this event the parameters used are;

\begin{enumerate}[(i)]
\item Identification of authorization level of the requesting officer (Must be the Permanent Sectary or authorized replacement)
\item Checking the availability of a workplan for a given request. 
\end{enumerate}

\subsubsection{Funds Release By Central Bank (BOU)}

\noindent This is the process of releasing funds to requesting entity on a periodic basis as per the budget arrangements. It involves transferring funds from the government account at the Central Bank (Bank of Uganda) to the respective accounts via electronic payment systems (EFTs).

\noindent The parameters used here are;

\begin{enumerate}[(i)]
\item Checking that the debited account is not in a dormant status(Dormant Accounts not to be debited).
\item Checking that the disbursed amount does not exceed the set limit for daily releases to ministries.
\end{enumerate}

\subsubsection{Emails and Phone Logs Events}

\noindent These are closely linked to the Funds Release event given that before money is released a confirmation email must be sent from the treasury endorsing the request and a phone call must be made from the Central Bank to the Permanent Secretary confirming that he/she authorized the funds request.

\noindent Therefore for these two events,the parameters are ;

\begin{enumerate}[(i)]
\item Availability of a confirmation email record from the treasury.
\item Availability of confirmation phone call record from Bank of Uganda to the Permanent Secretary.
\end{enumerate}

\subsubsection{Payments Event}

\noindent This is a record of how the funds were spent and may include payment details, account statements from the entity and suppliers accounts and also account opening details.A report of activities completed periodically must be availed to get value for money. Here the work done is verified against the budget or work plan submitted at the point of request for funds.

\noindent The parameters include;

\begin{enumerate}[(i)]
\item Checking if amount paid exceeds agreed amount in contract.
\item Checking that work being paid for was completed and not pending.
\end{enumerate}

\subsection{Tool Functionality/Operation}

\subsubsection{Client Request}

\noindent The process begins with a case file number being added for processing by the user. The user then defines the parameters to be queried by the CEP engine for the five events. The case file number ensures that only events for a particular case are generated from the database and forwarded to the engine for processing.

\begin{center}
\begin{figure}[h]
\centerbmp{14cm}{10cm}{pic1.bmp}
\caption{Case Input and Selection screen(Web Client)}

\end{figure}
\end{center}


\subsubsection{System Response}

\noindent On completion of event processing by the engine, events matching the defined criteria are displayed in tabular format to the web or mobile page and a PDF file is concurrently generated to a local location. Additionally, the total numeber of records and a percentage matching rate are computed and displayed at the top of the page.

\begin{center}
\begin{figure}[h]
\centerbmp{14cm}{8cm}{pic2.bmp}
\caption{Web View }

\end{figure}
\end{center}

\begin{center}
\begin{figure}[h]
\centerbmp{14cm}{3cm}{pic3.bmp}
\caption{Generated PDF files for multiple Case Files}

\end{figure}
\end{center}

\begin{center}
\begin{figure}[h]
\centerbmp{14cm}{10cm}{pic4.bmp}
\caption{PDF File Display}

\end{figure}
\end{center}

\subsubsection{Real-time Function}

\noindent The system displays in web view updated information based on the data in the CSV event files or underlying database. Any changes to the database or CSV files is automatically reflected in ten seconds on the web page using AJAX techniques embedded in the PrimeFaces server controls used.

\subsection{Test Results and Analysis}

\subsubsection{Events Generation from Database}

\noindent Events were successfully generated in CSV format for the case file number supplied by the investigator/user through the web or mobile interface. The files generated are placed in the project class path from where the processing CEP engine \cite{twentyfive} picks them for processing. The CSV files are only created for case files with existing records in the database for any of the five events(Funds Request, Funds Release,Email,PhoneLogs and Payments) supported by the system.

\noindent In a live production environment, the location of the event files may be managed differently taking into consideration the CEP engine \cite{twentyfive} and also the user needs. The file generation tests were  generally successful and quite fast give the use of JPA \cite{oracle}entity to query the database using its object mapping architecture.

\subsubsection{Defining Criteria and Processing Events}

\noindent The web and mobile interfaces provide the user with the functionality to determine what the CEP engine \cite{twentyfive} should query. The users were able to easily select values using the available GUI(Graphical User Interface) controls without any issues.

\noindent The processing speed was noted to be good though not as expected perhaps owing to the fact that system architecture is such that both generation events and event processing are launched by one command. This architecture may need to be reviewed in future to enable faster processing taking advantage of the CEP engine \cite{twentyfive} processing capabilities. It should also be noted that the tests carried out involved very few records and a computer with good specifications (3GHZ,8GB RAM, Core i7  Processor) and therefore processing was expected to be very quick.

\subsubsection{Output of Computed Results}

\noindent The output to the user is two-fold in PDF format as a file and a live display to the web or mobile page. It also supports a real-time element that relays results based on any changes in the incoming event files or database tables.

\noindent The PDF report file was found to be informative enough except for the not so appealing look. The web page view had a good feel and look with organized tabular display of results. The mobile had some challenges with adjusting to different mobile resolutions but this could have been corrected with more time availed. The reporting can be improved in the future by using graphs to indicate events from time to time as they happen.

\noindent Generally, the system was a success with at least the basic requirements working as expected. It is expected that with some improvement in the variety of events supported and information analysis, the tool will be of great importance to the Crime Investigation Department of the Uganda Police and other government agencies.

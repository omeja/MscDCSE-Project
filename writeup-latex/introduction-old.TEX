\section{Introduction}
\pagenumbering{arabic}
\subsection{Background}

\noindent Crime investigation world over is considered a difficult and laborious process that heavily relies on efficient crime analysis to ensure accurate and timely conclusions. This is not helped by the fact that law enforcement agencies deploy manual investigation processes and computerized systems that cannot quickly identify complex crime patterns.

\noindent In Uganda, the situation is no different with these agencies facing massive delays in crime solving and increasing numbers of sophisticated crimes, most of them of the organized. Every tens of thousands of cases are carried forward to the following year, uncompleted. As the usual circle of crime would dictate, fresh cases are reported every day, and, gradually, older cases left uncompleted lose the urgency they initially generated and, inadvertently, they die a natural death \cite{eleven}.

\noindent To avert this problem there is need to deploy sophisticated tools, technologies and resources that can enable crime investigators quickly reach reasonable conclusions by identifying patterns of behavior in criminals. In so doing, not only will crimes be investigated faster but some future crimes may be prevented based on recurring patterns identified. However, intelligence and law enforcement agencies are often faced with the dilemma of having too much data, which in effect makes too little value. On one hand, they have large volumes of raw data collected from multiple sources: phone records, bank accounts and transactions, vehicle sales and registration records, and surveillance reports. On the other hand, they lack sophisticated network analysis tools and techniques to utilize the data effectively and efficiently \cite{fifteen}.

\noindent In this project the phenomenon of Complex Event Processing (CEP) is used to detect patterns in crime related events using the Esper engine for high throughput and performance. CEP analyses low level events to produce a single complex event and has been successfully used in Stock Trading, Network Analysis and other areas.

\subsection{Statement of the problem}

\noindent Due to the manual nature of crime investigations in Uganda, case backlog remains a critical issue leading to delayed justice, unpunished criminals and of course tainting the image of the agencies. These manual methods often lead to fatigue, poor statistical analysis and the inability to solve crimes through pattern detection and analysis.

\noindent Criminals have become very intelligent due to the advancement of technology and therefore they conduct crimes in an untraceable manner. Majority of those crimes evolve in a long period of time making them even more difficult to predict. Therefore the rate of organized crime is on the rise most of which are orchestrated in vast geographical areas using these complex techniques.

\noindent Therefore, manual techniques of analyzing such data with a vast variation have resulted in lower productivity and ineffective utilization of manpower \cite{three}.There is need to develop a tool to quicken the investigation process by accurately guiding investigators in evidence analysis and also use recurrent patterns to prevent future crimes.

\subsection{Objectives}
\subsubsection{General Objective}

\noindent The objective of this project is to design, develop and deploy a crime investigation tool to guide investigators by detecting crime patterns and monitoring criminal activities. The tool will operate by filtering criminal activities as events and detecting predetermined patterns. This will not only reduce on time spent during investigations but also prevent some crimes from occurring by discovering recurring patterns. 

\subsubsection{Specific Objectives}
Specific objectives include:-
\begin{enumerate}
\item Design and build a crime investigation tool based on complex events with a processing engine and web-based visual interface.
\item Thoroughly validate and test the solution.
\item Deploy the solution in a live working environment.
 \end{enumerate}

\subsection{Scope}
The project covers general aspects of using complex event processing in applications but the implemented tool processes data from an existing information system and only focuses on crime patterns inherent in financial embezzlement related crimes. Document processing and forensics related actions are not covered in this project.

\subsection{Significance of Project}

\noindent The benefits of the project apply to the Police and the general public. These include:-
\begin{enumerate}
\item Improve efficiency by reducing manual operations and concentrating on analysis and prediction efforts hence also reducing on the time spent during investigations.  
\item Reduce on crime rates by preventing their occurrence through discovery of uniform patterns.

\end{enumerate}

\subsection{Key Contribution}

\noindent This project demonstrates the ability to use the idea of complex event processing in various areas of life by streaming different activities as incoming events and matching them against defined patterns.Given that complex event procesing is common in realtime domains, this project also shows this phenomenon in a non realtime situation(crime investigation after crime as occured).

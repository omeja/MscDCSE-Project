\input setbmp

\section{CHAPTER FIVE: Conclusion }

\noindent Crime investigations are an integral part of any law enforcement agency and if mismanaged can effect negatively on both government and the populace.By deploying techniques such as Complex Event Processing, Crime Investigation Systems can perform better and faster in analyzing data and discovering hidden patterns.Such tools will enable speedy investigations,overcome the labour shortage issue at the force and also improve public faith in the agencies and government at large.

\noindent Based on the tests carried out during the project, the crime investigation tool built was able to process records and accurately report events matching the defined query criteria.The project was successful although the input data formats were of a fixed type (CSV) and some few challenges.It is expected that with further developments and consultations the tool will be improved to cover all crimes,allow different file input formats as well as real-time monitoring of events and identify emerging patterns.

\subsection{Challenges}

\noindent The following challenges were faced during developement;

\begin{enumerate}[(i)]
\item Enabling the realtime monitoring mode given the complex architecture.
\item Representing crime data in a structured format to feed the engine.
\item Learning the Esper engine API which was time-consuming.
\item Interviewing some top Police officials who were rarely available.


\end{enumerate}

\subsection{Contribution}


\begin{enumerate}[(i)]
\item To computing, using CEP techniques and the esper engine to solve a problem in crime investigations which proved more effective than data mining techniques.
\item To citizens, enabling quick crime anlaysis leading to a good turn around time as well as managing small work force using technological initiatives.
\item To government and the Police,public confidence will be restored.


\end{enumerate}

\subsection{Recommendations}

\noindent The following recommendations were made ;

\begin{enumerate}[(i)]
\item Develop a well organized database to supply the tool engine with data and deploy it on a  country-wide Police network.
\item Agree and create case file numbers unique to every Police Station in the country.
\item Agree and arrange channels linking the system with other stakeholders' systems to provide straight-through communication e.g to the judiciary.
\item The Uganda Police needs to train investigators in computer-enabled investigation and equip them with neccesary ICT skills.
\end{enumerate}

\subsection{Future Work}

\noindent To improve the performance and usefullness of the tool, the following are suggested ;

\begin{enumerate}[(i)]
\item Extending the tool to detect new emerging patterns that can help in crime prevention based on available records.
\item Include a criminal monitoring module for continuous surveillance of criminals/areas based on known patterns.


\end{enumerate}

\input setbmp

\section {Appendices}


\subsection{Appendix A : Request for Permission and Approval Letters}

\begin{center}
\begin{figure}[h]
\centerbmp{14cm}{16cm}{a1.bmp}
\caption{Request for Permission Letter}

\end{figure}
\end{center}
\newpage
\begin{center}
\begin{figure}[h]
\centerbmp{14cm}{16cm}{a2.bmp}
\caption{Approval Letter from Uganda Police}

\end{figure}
\end{center}


\newpage
\subsection{Appendix B : Interview Guidelines and Questions }

\noindent This sections describes the guidelines used to carry out interviews and design questionnaires and some of the questions asked as well as responses from the interviewees.

\subsubsection{Guidelines}

\noindent The guidelines followed were related to the main goals of the project, determining whom to interview and also establishing relevant subgroups depending on their kind of work. The main goals of the interviews were :-

\begin{enumerate}[(i)]
\item Understanding the current crime investigation process from the point of the case being reported to time of sending the investigation report to the Director of Public Prosecutions (DPP).
\item Identifying crime entities involved and their associations.
\item Discovering the kind of data gathered during investigations and their corresponding sources and formats.
\item Studying known crime patterns related to financial crimes investigations.
\item Identifying any existing crime systems if any and  the kind of data stored.
\item Identifying expected outputs in terms of reports from the system.

\end{enumerate}

\noindent There was also the need to determine whom to interview.To do  this management was asked to provide details of officers responsible for criminal investigations and also guide about the departments specializing on financial crimes. In doing this, the roles of the officers were identified and a clear picture of their operations was understood. This also helped in gathering their thoughts about the current system and how it could be improved.

\noindent Another factor was detrmining the number of people to interview.It was important to determine the number of people to interview by clearly understanding the different roles played in the investigation process and also discovering the sections or departments that will be covered by the project.

\subsubsection{Interview Questions}

\noindent These were the interview questions asked during sessions with the director of investigations, investigative police officers and Information Technology experts at the force.

\begin{enumerate}[(i)]
\item Please describe the current investigation process for the crime of "Causing financial loss to government" ?
\item In such a case who are the key players involved and what kind of records do you get from them for crime analysis?
\item Are the above records stored in an electronic database in house?
\item How is the above data used to unearth criminal activities and what known pattern in used as reference?
\item Are the investigative officers skilled in ICT procedures and to what level?
\item What challenges are faced in the current system and how do they impact government and the citizens?
\item Which officers are knowledgeable about financial crime investigations and under which departments are the attached?
\item Are all Police Stations code-named uniquely and What is the reference number format?
\item What kind of reports are generated after the investigations are completed and in what format?
\item Which departments will use the new system?
\item Who has got access to investigation data?
\end{enumerate}





	 


































